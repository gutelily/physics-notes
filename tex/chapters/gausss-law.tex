
\chapter{Gauss's law}
Given an electric field $\mathbf{E}(\mathbf{r})$ and a closed surface $A$, according to Gauss's law
\be\label{gauss-law}
\Phi_E = \oiint_A \mathbf{E} \cdot \mathrm{d}\mathbf{A} =\frac{Q_{encl}}{\epsilon_0}
\ee
where $Q_{encl}$ is the total charge enclosed in $A$, $\mathrm{d}\mathbf{A}=\mathrm{d}A\uvec{A}$ is an infinitesimal area $\mathrm{d}A$ with direction $\uvec{A}$, the latter defined as the normal of the surface at the point under consideration.

\section{Calculation of electric field using Gauss's law}
\subsection{Point-like charge}
Given a particle with charge $q$ in empty space, due to the spherical symmetry of the system, the electric field $\mathbf{E}(r,\theta,\phi)$ cannot depend on $\theta$ and $\phi$. In addition, the electric field must be in the direction of $\uvec{r}$. Hence, we can write
\be\label{spherically-symmetric-e-field}
\mathbf{E} = E_r(r) \cdot \uvec{r}
\ee

By replacing \ref{spherically-symmetric-e-field} in \ref{gauss-law}, we can write
\be\label{spherically-symmetric-flux}
\oiint_A E_r(r) \cdot \uvec{r} \cdot \mathrm{d}\mathbf{A} =\frac{Q_{encl}}{\epsilon_0}
\ee

Considering a sphere centered at the location of the charged particle, \ref{spherically-symmetric-flux} becomes, in spherical coordinates,
\be\label{spherically-symmetric-flux-2}
\oiint_A E_r(r) \cdot \uvec{r} \cdot r^{2}\sin(\phi)\mathrm{d}\theta\mathrm{d}\phi\uvec{r} =\frac{q}{\epsilon_0}
\ee
from which
\be
\mathbf{E}=\frac{q}{4\pi\epsilon_0 r^{2}}\uvec{r}
\ee
which matches \ref{eq:electric-field-final}.


\subsection{Infinitely long, straight wire}
Given a wire with charge density per unit of length $\lambda$ in empty space, due to the cylindrical symmetry of the system, the electric field $\mathbf{E}(r,\theta,z)$ cannot depend on $\theta$ and $z$. In addition, the electric field must be in the direction of $\uvec{\rho}$. Hence, we can write
\be\label{cylindrically-symmetric-e-field}
\mathbf{E} = E_{\rho}(\rho) \cdot \uvec{r}
\ee

By replacing \ref{cylindrically-symmetric-e-field} in \ref{gauss-law}, we can write
\be\label{cylindrically-symmetric-flux}
\oiint_A E_{\rho}(\rho) \cdot \uvec{r} \cdot \mathrm{d}\mathbf{A} =\frac{Q_{encl}}{\epsilon_0}
\ee

where $\mathrm{d}\mathbf{A} =  \rho\uvec{r}\mathrm{d}\mathbf{\theta} \mathrm{d}\mathbf{z}$ is an infinitesimal area with the direction of $\uvec{r}$.

Considering a cylinder centered at the location of the charged wire, \ref{cylindrically-symmetric-flux} becomes,
\be\label{cylindrically-symmetric-flux-2}
\oiint_A E_{\rho}(\rho) \cdot \rho \mathrm{d}\theta\mathrm{d}z \cdot\uvec{r} =\frac{Q_{encl}}{\epsilon_0}
\ee
from which
\be
\mathbf{E}=\frac{\lambda}{2\pi\epsilon_0 r}\uvec{r}
\ee



\subsection{Infinite plane}
An infinite plane with charge density per unit of area $\sigma$ in empty space, located on the x-y plane is given. Due to the symmetry of a system, the electric field is constant and in the direction of the z-axis.


\be\label{infinite-plane-e-field}
\mathbf{E} = E_{z}(z) \cdot \uvec{z} = E_z \cdot \frac{z}{|z|}\uvec{z}
\ee

By replacing \ref{infinite-plane-e-field} in \ref{gauss-law}, we can write
\be\label{infinite-plane-flux}
\oiint_A E_{z}(z) \cdot \uvec{z} \cdot \mathrm{d}\mathbf{A} =\frac{Q_{encl}}{\epsilon_0}
\ee
where $\mathrm{d}\mathbf{A} = \mathrm{d}x \mathrm{d}y$ is infinitesimal area with the direction normal to the surface and parallel to the z-axis.

Considering a parallelepiped centered at the location of the charged plane, the only contributions to the flux are those that are parallel to z-axis,

% add info about all faces and 6 integrlas -- everything in your notes + needed descriptions

\be\label{}
\oiint_A E_z \cdot \frac{z}{|z|}\uvec{z} \cdot \mathrm{d}x\mathrm{d}y\uvec{z} + \oiint_A E_z \cdot \frac{-z}{|z|}\uvec{z} \cdot \mathrm{d}x\mathrm{d}y\uvec{z} = \frac{Q}{\epsilon_0}
\ee
from which
\be
\mathbf{E}=\frac{\sigma}{2\epsilon_0} \cdot \frac{z}{|z|} \uvec{r}
\ee
