\section{Introduction}
Let $q$ and $Q$ be the charges of two point-like particles, located at positions $\mathbf{r}_{q}$ and $\mathbf{r}_{Q}$, respectively
(see Fig. \ref{fig:two-point-charges}).
\begin{figure}
	\centering
\tdplotsetmaincoords{60}{110}
\begin{tikzpicture}[scale=2,tdplot_main_coords]\tikzset{>=latex}
  
  % VARIABLES
  \def\xq{2.6}
  \def\yq{3}
  \def\zq{2.8}
  \def\xQ{2.4}
  \def\yQ{0.2}
  \def\zQ{2.4}
  \def\s{0.3}
  \coordinate (r_q) at (\xq,\yq,\zq);
  \coordinate (r_Q) at (\xQ,\yQ,\zQ);
  % AXES
  \coordinate (O) at (0,0,0);
  \axesthree{2}
  
  % VECTOR
  \draw[->](O) -- node[below right] {$\vec{r}_{q}$} (r_q) ;
  \draw[->](O) -- node[left] {$\vec{r}_{Q}$} (r_Q);
  \node at (r_q)[circle,fill,inner sep=1.5pt]{}; 
  \node at (r_Q)[circle,fill,inner sep=1.5pt]{};
  \node at (r_q)[above]{$q$};
  \node at (r_Q)[above]{$Q$}; 
  \draw[->](r_Q) --  node[above left] {$\vec{r}$} (r_q);
  \draw[->][thick](r_q) -- +($(\xq*\s, \yq*\s,\zq*\s)-(\xQ*\s, \yQ*\s,\zQ*\s)$) node[above] {$\vec{F}_{q,Q}$};

\end{tikzpicture}
	\caption{Electric force exerted by $Q$ on $q$.} \label{fig:two-point-charges}
\end{figure}
According to Coulomb's law, the force experienced by $q$ due to $Q$ is:
\be\label{eq:culombs-law}
\mathbf{F}_{q,Q}(\mathbf{r}_q)=\frac{qQ}{4\pi\epsilon_0}\frac{\mathbf{r}_{q}-\mathbf{r}_{Q}}{\norm{\mathbf{r}_{q}-\mathbf{r}_{Q}}^{3/2}}
\ee
It is clear that the quantity $\mathbf{F}_{q,Q}/q$ does not depend on $q$. This is true under the assumption that the presence of $q$ does not influence the mathematical model provided in \ref{eq:culombs-law}, in particular, if $q$ is infinitely small. Under this assumption, a formal definition of field is provided in by the following.
\be\label{eq:electric-field}
\mathbf{E}(\mathbf{r}_{q})=\lim_{q \to 0} \frac{\mathbf{F}_{Q,q}}{q}
\ee
At any point $\mathbf{r}_{q}$, \ref{eq:electric-field} is the electric field generated by a point-like particle with charge $Q$, located at $\mathbf{r}_{Q}$. From \ref{eq:culombs-law}, it is clear that the relative location $\mathbf{r}_{q}-\mathbf{r}_{Q}$ governs the field calculation. It is therefore convenient to define the vector $\mathbf{r}=\mathbf{r}_{q}-\mathbf{r}_{Q}$ and write \ref{eq:electric-field} as 
\be\label{eq:electric-field-final}
\mathbf{E}(\mathbf{r})=\frac{1}{4\pi\epsilon_0}\frac{Q}{r^{2}}\uvec{r}
\ee
See Fig. \ref{fig:electric-field-point-charge}.
\begin{figure}
	\centering
% 3D AXIS with spherical coordinates
\tdplotsetmaincoords{60}{110}
\begin{tikzpicture}[scale=2,tdplot_main_coords]\tikzset{>=latex}
  
  % VARIABLES
  \def\xq{2.6}
  \def\yq{3}
  \def\zq{2.8}
  \def\xQ{2.4}
  \def\yQ{0.2}
  \def\zQ{2.4}
  \def\s{0.3}
  \coordinate (r_q) at (\xq,\yq,\zq);
  \coordinate (r_Q) at (\xQ,\yQ,\zQ);
  % AXES
  \coordinate (O) at (0,0,0);
  \axesthree{2}
  
  % VECTORS
  \draw[->](O) -- node[below right] {$\vec{r}_{q}$} (r_q) ;
  \draw[->](O) -- node[left] {$\vec{r}_{Q}$} (r_Q);
  \draw[->](r_Q) --  node[above] {$\vec{r}$} (r_q);
  \node at (r_Q)[circle,fill,inner sep=1.5pt]{};
  \node at (r_Q)[above]{$Q$};
  \draw[->][thick](r_q) -- +($(\xq*\s, \yq*\s,\zq*\s)-(\xQ*\s, \yQ*\s,\zQ*\s)$) node[above] {$\vec{E}(\vec{r})$};

\end{tikzpicture}
	\caption{Electric field generated by a point-like charge $Q$.} \label{fig:electric-field-point-charge}
\end{figure}