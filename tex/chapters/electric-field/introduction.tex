\section{Introduction}
Let $q$ and $Q$ be the charges of two point-like particles, located at positions $\mathbf{r}_{q}$ and $\mathbf{r}_{Q}$, respectively.
According to Coulomb's law, the force experienced by $q$ due to $Q$ is:
\be\label{eq:culombs-law}
\mathbf{F}_{q,Q}(\mathbf{r}_q)=\frac{qQ}{4\pi\epsilon_0}\frac{\mathbf{r}_{q}-\mathbf{r}_{Q}}{\norm{\mathbf{r}_{q}-\mathbf{r}_{Q}}^{3/2}}
\ee
It is clear that the quantity $\mathbf{F}_{q,Q}/q$ does not depend on $q$. This is true under the assumption that the presence of $q$ does not influence the mathematical model provided in \ref{eq:culombs-law}, in particular, if $q$ is infinitely small. Under this assumption, a formal definition of field is provided in by the following.
\be\label{eq:electric-field}
\mathbf{E}(\mathbf{r}_{q})=\lim_{q \to 0} \frac{\mathbf{F}_{Q,q}}{q}
\ee
At any point $\mathbf{r}_{q}$, \ref{eq:electric-field} is the electric field generated by a point-like particle with charge $Q$, located at $\mathbf{r}_{Q}$. From \ref{eq:culombs-law}, it is clear that the relative location $\mathbf{r}_{q}-\mathbf{r}_{Q}$ governs the filed calculation. It is therefore convenient to define the vector $\mathbf{r}=\mathbf{r}_{q}-\mathbf{r}_{Q}$ and write \ref{eq:electric-field} as 
\be\label{eq:electric-field-final}
\mathbf{E}(\mathbf{r})=\frac{1}{4\pi\epsilon_0}\frac{Q}{r^{2}}\uvec{r}
\ee