\section{Electric field generated by a straight, infinitely long wire}
\begin{figure}[!htbp]
	\centering
\tdplotsetmaincoords{60}{110}
\begin{tikzpicture}[scale=2,tdplot_main_coords]\tikzset{>=latex}
  
  % VARIABLES
  \def\xq{1.6}
  \def\yq{1.6}
  \def\zq{2.3}
  \def\xQ{0}
  \def\yQ{0}
  \def\zQ{1.5}
  \def\s{0.3}
  \coordinate (r_q) at (\xq,\yq,\zq);
  \coordinate (r_Q) at (\xQ,\yQ,\zQ);
  % AXES
  \coordinate (O) at (0,0,0);
  \axesthree{3}
  
  % VECTOR
  \draw[->](O) -- node[below] {$\vec{r}_{0}$} (r_q);
  \draw[->](0,0,\zq) -- node[left] {$\ell$} (0,0,\zQ) node[circle,fill,inner sep=1.5pt]{};
  \draw[->](O) -- (0,0,\zq) node[left] {$z_{0}$};
  \draw[->](O) -- node[left]{$z_{0}+\ell$} (0,0,\zQ);
  \draw[->](r_Q) --  node[above left] {$\vec{r}$} (r_q);
  \draw[->][thick](r_q) -- +($(\xq*\s, \yq*\s,\zq*\s)-(\xQ*\s, \yQ*\s,\zQ*\s)$) node[above] {$\total{\vec{E}}$};

\end{tikzpicture}
	\caption{Infinitely long wire with uniform charge density.} \label{fig:infinite-wire}
\end{figure}
Let $\lambda$ be the electric charge density per unit of length of a straight, infinitely long wire, located along the $z$-axis (see \ref{fig:infinite-wire}). At any point $\mathbf{r}_{0}=(x_{0},y_{0},z_{0})$, the contribution to the total electric field given by an infinitesimal portion of it $\total{\ell}$, located in space at position $(0,0,\ell+z_{0})$, is given by \ref{eq:electric-field-final} as follows:
\be\label{eq:infinitesimal-electric-field-wire}
\total{\mathbf{E}(\mathbf{r})}=\frac{1}{4\pi\epsilon_0}\frac{\lambda\total{\ell}}{r^{2}}\uvec{r}
\ee
By defining $d=\sqrt{x_{0}^{2}+y_{0}^{2}}$, being the distance between $\mathbf{r}$ and the closest point on the wire, the total electric field becomes
\be\label{eq:total-electric-field-wire}
\mathbf{E}(\mathbf{r})=\frac{\lambda}{4\pi\epsilon_0}\int_{-\infty}^{+\infty}\frac{\total{\ell}}{(d^{2}+\ell^{2})^{3/2}}(x_{0},y_{0},-\ell)
\ee
Notice that the $z$-component of the integral is an odd function, integrated between a symmetric interval with respect to the origin. Therefore, the total contribution to the field in the $z$-direction must be zero. Hence, we can simplify \ref{eq:total-electric-field-wire} as follows:
\be\label{eq:total-electric-field-wire-simplified}
\mathbf{E}(\mathbf{r})=(x_{0},y_{0},0) \frac{\lambda}{4\pi\epsilon_0}\int_{-\infty}^{+\infty}\frac{\total{\ell}}{(d^{2}+\ell^{2})^{3/2}}
\ee
By defining $\phi$ such that $\ell=d\tan(\phi)$, we obtain $\total{\ell}=d/\cos^{2}(\phi)$ and \ref{eq:total-electric-field-wire-simplified} becomes
\be\label{eq:total-electric-field-wire-using-phi}
\mathbf{E}(\mathbf{r})=(x_{0},y_{0},0) \frac{\lambda}{4\pi\epsilon_0}\int_{-\pi/2}^{+\pi/2}\frac{\cos(\phi)\total{\phi}}{d^{3}}
\ee
By definition of $d$, the quantity $(x_{0},y_{0},0)/d$ is a unit vector which we can call $\uvec{d}$. This vector represents the direction of the total field which always points towards $\mathbf{r}_{0}$ from the closest point along the wire. As a result, we can write
\be\label{eq:total-electric-field-wire-final}
\mathbf{E}(\mathbf{r})=\frac{\lambda}{2\pi\epsilon_{0}d}\uvec{d}
\ee
In cylindrical coordinates $(\rho,\theta,z)$, the same field reads
\be
\mathbf{E}(\mathbf{r})=\frac{\lambda}{2\pi\epsilon_{0}\rho}\uvec{\rho}=E_{\rho}(\rho)\uvec{\rho}
\ee
showing that the field always points in the radial direction and depends only on the length of the radius.