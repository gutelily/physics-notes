\section{Electric field generated by an infinite plane}
Let $\sigma$ be the electric charge density per unit of area of an infinite plane, located on the $XY$-plane of a coordinate system $(x,y,z)$. In accordance with \ref{eq:electric-field-final}, the infinitesimal contribution to the electric filed given by an infinitesimal area $\total{A}=\total{x}\total{y}$ is
\be
\total{\mathbf{E}(\mathbf{r})}=\frac{\sigma}{4\pi\epsilon_0}\frac{\total{x}\total{y}}{r^{2}}\uvec{r}
\ee
At a point $\mathbf{r}_{0}=(x_{0},y_{0},z_{0})$, the infinitesimal area $dxdy$ located at $(x,y,0)$ can be expressed as $(\rho\cos(\theta)+x_{0},\rho\sin(\theta)+y_{0},0)$, so that, by changing $\theta$, a circle centered at $(x_{0},y_{0},0)$ is described. The vector $\mathbf{r}$ becomes $(-\rho\cos(\theta),-\rho\sin(\theta),z_{0})$.
Using the standard transformations form Euclidean to cylindrical coordinates, $\total{x}\total{y}$ becomes $\rho\total{\rho}\total{\theta}$, $r^{2}=\rho^{2}+z_{0}^{2}$, and the total filed is given by 
\be\label{eq:total-electric-field-plane}
\mathbf{E}(\mathbf{r})=\frac{\sigma}{4\pi\epsilon_0}\int_{0}^{2\pi}\total{\theta}\int_{0}^{+\infty}\total{\rho}\frac{\rho}{(\rho^{2}+z_{0}^{2})^{3/2}}(-\rho\cos(\theta),-\rho\sin(\theta),z_{0})
\ee
Notice that integrals of $\cos(\theta)$ and $\sin(\theta)$ between $0$ and $2\pi$ vanish, and hence the $x$- and $y$-coordinates of the total field must be $0$. We can rewrite \ref{eq:total-electric-field-plane}
\be\label{eq:total-electric-field-plane-simplified}
\mathbf{E}(\mathbf{r})=(0,0,z_{0})\frac{\sigma}{4\pi\epsilon_0}\int_{0}^{2\pi}\total{\theta}\int_{0}^{+\infty}\total{\rho}\frac{\rho}{(\rho^{2}+z_{0}^{2})^{3/2}}
\ee
The integral over $\theta$ equals $2\pi$. The integral over $\rho$ is $-1/(\rho^{2}+z_{0}^{2})^{1/2}$ calculated between $0$ and $+\infty$, hence returning
\be\label{eq:total-electric-field-plane-final}
\mathbf{E}(\mathbf{r})=(0,0,\frac{\sign(z)\sigma}{2\epsilon_0})=E_{z}(\sign(z))\uvec{z}
\ee
showing that strength of the field is constant everywhere and its direction is always pointing away from the plane.