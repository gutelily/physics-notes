\section{Dialectric ball}
Given a dialectric ball with radius $R_{0}$ and uniform charge density per unit of volume $\rho$. Due to
the spherical symmetry of the system the electric field $E(r,\theta,\phi)$ cannot depend on $\theta$
and $\phi$, so the electric field must be in the direction of $r$.
\be\label{dialectric-ball-e-field}
\mathbf{E} = E_r(r) \uvec{r}
\ee

By replacing \ref{dialectric-ball-e-field} in \ref{gauss-law}, we can write
\be\label{ball-surface-flux}
\oiint_A E_r(r) \uvec{r} \cdot \mathrm{d}\mathbf{A} =\frac{Q_{encl}}{\epsilon_0}
\ee
where $\total{\mathbf{A}}=R_{0}^2sin(\phi)\total{\theta}\total{\phi}\uvec{r}$ is an infinitesimal 
area with the direction of $\uvec{r}$, so that \ref{ball-surface-flux} becomes, in spherical coordinates,
\be\label{ball-surface-flux-2}
\oiint_A E_r(r)\uvec{r} \cdot \uvec{r}R_{0}^{2}\sin(\phi)\mathrm{d}\theta\mathrm{d}\phi =\frac{Q_{encl}}{\epsilon_0}
\ee
from which, $Q_{encl}=\frac{4}{3}\rho\pi r^3$. On the surface or outside ball $r=R$ when $R_{0}>=R$  and inside the ball $r=R_0$ when $R_{0}<R$. so we have.
Electric field calculated inside the ball

Electric field calculated on the surface or outside
\be
\mathbf{E}=\frac{q}{4\pi\epsilon_0 r^{2}}\uvec{r}
\ee
which matches \ref{eq:electric-field-final}.
