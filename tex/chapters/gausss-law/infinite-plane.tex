\section{Infinite plane}
An infinite plane with charge density per unit of area $\sigma$ in empty space, located on the $XY$-plane is given. Due to the symmetry of the 
system, the electric field strength is constant and points away from the plane, orthogonally to the plane.


\be\label{infinite-plane-e-field}
\vec{E} = E_{z}(z)\uvec{z} = E_{z}(\sign(z))\uvec{z}
\ee

By replacing \ref{infinite-plane-e-field} in \ref{gauss-law}, we can write
\be\label{infinite-plane-flux}
\oiint_A E_{z}(\sign(z))\uvec{z} \cdot \total{\vec{A}} =\frac{Q_{encl}}{\epsilon_0}
\ee

Considering a parallelepiped whose faces are on the planes $x=x_{0}$, $x=-x_{0}$, $y=y_{0}$, $y=-y_{0}$, $z=z_{0}$, $z=-z_{0}$, where $x_{0}$, $y_{0}$, and $z_{0}$ are positive real numbers. The total flux equals to the sum of the fluxes of each face. 

\be\label{eq:parallelepiped-fluxes}
\Phi_{E_{tot}}=\Phi_{x_{+}}+\Phi_{x_{-}}+\Phi_{y_{+}}+\Phi_{y_{-}}+\Phi_{z_{+}}+\Phi_{z_{-}}
\ee

The subscripts in \ref{eq:parallelepiped-fluxes} denote the directions of the face normals, e.g.
$x_{+}$ stands for "positive $x$-direction", $x_{-}$ stands for "negative $x$-direction", and so on.

The flux is calculated as the dot product between $\vec{E}$ and $\total{\vec{A}}$, but because electric field is in the $z$-direction, the contributions to the flux in $x$- and $y$-directions are zero. Therefore, we have
\be\label{}
\iint_{top} E_z\sign(z_{0})\uvec{z} \cdot\uvec{z} \total{x}\total{y}+
\iint_{bottom} E_z\sign(-z_{0})\uvec{z} \cdot(-\uvec{z}) \total{x}\total{y}
=\frac{4\sigma x_{0}y_{0}}{\epsilon_0}
\ee
from which
\be
\vec{E}=\frac{\sigma}{2\epsilon_0}\sign(z)\uvec{z}
\ee
