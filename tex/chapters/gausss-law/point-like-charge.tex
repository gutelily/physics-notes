\section{Point-like charge}
Given a particle with charge $q$ in empty space, due to the spherical symmetry of the system, the electric field $\vec{E}(r,\theta,\phi)$ 
cannot depend on $\theta$ and $\phi$. In addition, the electric field must be in the direction of $\uvec{r}$. Hence, we can write
\be\label{spherically-symmetric-e-field}
\vec{E} = E_r(r) \uvec{r}
\ee

By replacing \ref{spherically-symmetric-e-field} in \ref{gauss-law}, we can write
\be\label{spherically-symmetric-flux}
\oiint_A E_r(r) \uvec{r} \cdot \mathrm{d}\vec{A} =\frac{Q_{encl}}{\epsilon_0}
\ee

Considering a sphere centered at the location of the charged particle, \ref{spherically-symmetric-flux} becomes, in spherical coordinates,
\be\label{spherically-symmetric-flux-2}
\oiint_A E_r(r) \uvec{r} \cdot \uvec{r}r^{2}\sin(\phi)\mathrm{d}\theta\mathrm{d}\phi =\frac{q}{\epsilon_0}
\ee
from which, due to the fact that $\uvec{n}=\uvec{r}$
\be
\vec{E}=\frac{q}{4\pi\epsilon_0 r^{2}}\uvec{r}
\ee
which matches \ref{eq:electric-field-final}.
