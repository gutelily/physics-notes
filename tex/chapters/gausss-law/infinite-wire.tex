\section{Infinitely long, straight wire}
Given a wire with charge density per unit of length $\lambda$ in empty space, due to the cylindrical symmetry of the system, the electric 
field $\mathbf{E}(r,\theta,z)$ cannot depend on $\theta$ and $z$. In addition, the electric field must be in the direction of $\uvec{\rho}$. 
Hence, we can write
\be\label{cylindrically-symmetric-e-field}
\mathbf{E} = E_{\rho}(\rho) \cdot \uvec{r}
\ee

By replacing \ref{cylindrically-symmetric-e-field} in \ref{gauss-law}, we can write
\be\label{cylindrically-symmetric-flux}
\oiint_A E_{\rho}(\rho) \cdot \uvec{r} \cdot \mathrm{d}\mathbf{A} =\frac{Q_{encl}}{\epsilon_0}
\ee

where $\mathrm{d}\mathbf{A} =  \rho\uvec{r}\mathrm{d}\mathbf{\theta} \mathrm{d}\mathbf{z}$ is an infinitesimal area with the direction of $\uvec{r}$.

Considering a cylinder centered at the location of the charged wire, \ref{cylindrically-symmetric-flux} becomes,
\be\label{cylindrically-symmetric-flux-2}
\oiint_A E_{\rho}(\rho) \cdot \rho \mathrm{d}\theta\mathrm{d}z \cdot\uvec{r} =\frac{Q_{encl}}{\epsilon_0}
\ee
from which
\be
\mathbf{E}=\frac{\lambda}{2\pi\epsilon_0 r}\uvec{r}
\ee
