\chapter{Electric field}
\section{Introduction}
Let $q$ and $Q$ be the charges of two point-like particles, located at positions $\mathbf{r}$ and $\mathbf{R}$, respectively.
According to Coulomb's law, the force experienced by $q$ due to $Q$ is:
\be\label{eq:culombs-law}
\mathbf{F_{Q,q}}=\frac{qQ}{4\pi\epsilon_0}\frac{\mathbf{r}-\mathbf{R}}{\norm{\mathbf{r}-\mathbf{R}}^{3/2}}
\ee
It is clear that the quantity $\mathbf{F}_{1,2}/q_{1}$ does not depend on $q_{1}$. This is true under the assumption that the presence of $q_{1}$ does not influence the mathematical model provided in \ref{eq:culombs-law}. This assumption is valid if $q_{1}$ is assumed to be infinitely small. Under such an assumption, a formal definition of field is provided in by the following.

\be\label{eq:electric-field}
\mathbf{E}(\mathbf{r})=\lim_{q \to 0} \frac{\mathbf{F}_{Q,q}}{q}
\ee

Equation \ref{eq:electric-field} is, at any point $\mathbf{r}$ the electric field generated by a point-like particle with charge $Q$, located at $\mathbf{R}$.
In a compact form, \ref{eq:electric-field} can be written as 
\be\label{eq:compact-electric-field}
\mathbf{E}(\mathbf{d})=\frac{1}{4\pi\epsilon_0}\frac{Q}{d^{2}}\uvec{d}
\ee
where $\mathbf{d}=\mathbf{\mathbf{r}-\mathbf{R}}$, $d=\norm{\mathbf{d}}$, and $\uvec{d}=\mathbf{d}/d$.

\section{Electric field generated by a straight, infinitely long wire}
Let $\lambda$ be the charge density per unit of length of a straight, infinitely long wire, located along the $z$ axis. At any point $\mathbf{r}$, the contribution to the total electric field give by an infinitesimal portion of it $\total{\ell}$, located in space at position $(0,0,\ell)$, is given by \ref{eq:compact-electric-field} as follows:
\be\label{eq:infinitesimal-electric-field-wire}
\total{\mathbf{E}(\mathbf{r})}=\frac{1}{4\pi\epsilon_0}\frac{\lambda\total{\ell}}{r^{2}}\uvec{r}
\ee
By defining $d=\sqrt{x^{2}+y^{2}}$, being the distance between $\mathbf{r}$ and the closest point on the wire, and $\phi$ the angle such that $d\tan(\phi)=\ell$, the total electric field becomes\dots
\be\label{eq:total-electric-field-wire}
\mathbf{E}(\mathbf{r})=\frac{\lambda}{4\pi\epsilon_0}\int_{-\infty}^{+\infty}\frac{\total{\ell}}{(d^{2}+\ell^{2})^{3/2}}(x,y,-\ell)
\ee
Notice that the $z$-component of the integral is an odd function, integrated between a symmetric interval with respect to the origin. Therefore, the total contribution to the field in the $z$-direction must be zero. Hence, we can simplify \ref{eq:total-electric-field-wire} as follows:
\be\label{eq:total-electric-field-wire-simplified}
\mathbf{E}(\mathbf{r})=(x,y,0) \frac{\lambda}{4\pi\epsilon_0}\int_{-\infty}^{+\infty}\frac{\total{\ell}}{(d^{2}+\ell^{2})^{3/2}}
\ee
By defining $\phi$ such that $\ell=d\tan(\phi)$, we obtain $\total{\ell}=d/\cos^{2}(\phi)$ and \ref{eq:total-electric-field-wire-simplified} becomes
\be\label{eq:total-electric-field-wire-using-phi}
\mathbf{E}(\mathbf{r})=(x,y,0) \frac{\lambda}{4\pi\epsilon_0}\int_{-\pi/2}^{+\pi/2}\frac{\cos(\phi)\total{\phi}}{d^{3}}
\ee
By definition of $d$, the quantity $(x,y,0)/d^{2}$ is a unit vector which we can call $\uvec{d}$. This vector represents the direction of the total field which always points towards $\mathbf{r}$ from the closest point along the wire. As a result, we can write
\be\label{eq:total-electric-field-wire-final}
\mathbf{E}(\mathbf{r})=\frac{\lambda}{2\pi\epsilon_{0}d}\uvec{d}
\ee
In cylindrical coordinates $(\rho,\theta,z)$, the same field reads
\be
\mathbf{E}(\mathbf{r})=\frac{\lambda}{2\pi\epsilon_{0}\rho}\uvec{\rho}=E_{\rho}(\rho)\uvec{\rho}
\ee
showing that the filed always points in the radial direction and depends only on the radius.