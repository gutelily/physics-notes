\chapter{Electric field}
\section{Introduction}
Let $q$ and $Q$ be the charges of two point-like particles, located at positions $\mathbf{r}$ and $\mathbf{R}$, respectively.
According to Coulomb's law, the force experienced by $q$ due to $Q$ is:
\be\label{eq:culombs-law}
\mathbf{F_{Q,q}}=\frac{qQ}{4\pi\epsilon_0}\frac{\mathbf{r}-\mathbf{R}}{\norm{\mathbf{r}-\mathbf{R}}^{3/2}}
\ee
It is clear that the quantity $\mathbf{F}_{1,2}/q_{1}$ does not depend on $q_{1}$. This is true under the assumption that the presence of $q_{1}$ does not influence the mathematical model provided in \ref{eq:culombs-law}. This assumption is valid if $q_{1}$ is assumed to be infinitely small. Under such an assumption, a formal definition of field is provided in by the following.

\be\label{eq:electric-field}
\mathbf{E(\mathbf{r})}=\lim_{q \to 0} \frac{\mathbf{F}_{Q,q}}{q}
\ee

Equation \ref{eq:electric-field} is, at any point $\mathbf{r}$ the electric field generated by a point-like particle with charge $Q$, located at $\mathbf{R}$.
In a compact form, \ref{eq:electric-field} can be written as 
\be
\mathbf{E(\mathbf{d})}=\frac{1}{4\pi\epsilon_0}\frac{Q}{d^{2}}\uvec{d}
\ee
where $\mathbf{d}=\mathbf{\mathbf{r}-\mathbf{R}}$, $d=\norm{\mathbf{d}}$, and $\uvec{d}=\mathbf{d}/d$.